\documentclass{llncs}
\usepackage{pythonhighlight}
%
\begin{document}
	
	
\title{Informe del Proyecto Final de Sistemas de Recuperaci\'on de Informaci\'on}
\author{Jes\'us Aldair Alfonso P\'erez C-312 \and Mauro Jos\'e Bolado Vizoso C-311}
\institute{Universidad de la Habana}


\maketitle

\begin{abstract}
	
	En este documento se describe el proceso seguido para la construcción de un sistema de recuperación de información que es utilizado para recuperar docuemntos mediante una consulta. El sistema consta de tres modelos, el conocido Modelo Vectorial, el Modelo Probabil\'istico y el Modelo Booleano Extendido. Se analiza primeramente el problema de la recuperación de información y la  evaluación sobre colecciones de pruebas. Seguidamente se describe el procesado léxico realizado sobre el contenido de documentos y consultas.
	
\end{abstract}

\section{Introducci\'on}

El problema de la recuperaci\'on de informaci\'on se ha venido presentando desde hace muchos a\~nos. Para darle soluci\'on a este problema se han desarrollado diferentes m\'etodos de recuperaci\'on de informaci\'on que han ido evolucionando con el paso del tiempo. Muchos cient\'ificos han trabajo a lo largo de los a\~nos para darle soluci\'on de una manera m\'as eficiente a este problema.\\

Las b\'usquedas en la web han sido una de las herramientas m\'as utilizadas por los usuarios desde que se cre\'o la web.\\

\subsection{Objetivo}
	El objetivo de este trabajo es el de realizar un sistema de recuperaci\'on de informaci\'on 
	que permita a los usuarios realizar b\'usquedas en un conjunto de documentos, 
	y obtener como resultado una lista de documentos ordenados por relevancia.\\

\subsection{Rese\~na Hist\'orica}
	El problema de la recuperaci\'on de informaci\'on se ha venido presentando desde hace
	muchos a\~nos. Para darle soluci\'on a este problema se han desarrollado diferentes
	m\'etodos de recuperaci\'on de informaci\'on que han ido evolucionando con el paso del
	tiempo. Muchos cient\'ificos han trabajo a lo largo de los a\~nos para darle soluci\'on
	de una manera m\'as eficiente a este problema.\\

	Las b\'usquedas en la web han sido una de las herramientas m\'as utilizadas por los usuarios
	desde que se cre\'o la web.\\

\subsection{Descripci\'on}
	El sistema de recuperaci\'on de informaci\'on que se desarrollar\'a en este trabajo
	ser\'ia un sistema de recuperaci\'on de informaci\'on basado en tres m\'etodos de
	recuperaci\'on de informaci\'on:\\

	\begin{itemize}
		\item M\'etodo de recuperaci\'on de informaci\'on basado en el modelo vectorial.
		\item M\'etodo de recuperaci\'on de informaci\'on basado en el modelo probabil\'istico.
		\item M\'etodo de recuperaci\'on de informaci\'on basado en el modelo booleano extendido.
	\end{itemize}
	
	Se le permite al usuario decidir si desea utilizar cualquiera de los tres m\'etodos
	para realizar la recuperaci\'on de informaci\'on.\\

	El usuario debe proporcionar una colecci\'on de documentos, y una consulta de b\'usqueda.\\

	\subsection{Modelo Vectorial}
	El modelo vectorial es un modelo de recuperaci\'on de informaci\'on. Se basa en la
	representaci\'on de los documentos y las consultas como vectores en un espacio vectorial de 
	\textit{n}-dimensiones, donde \textit{n} es el n\'umero de t\'erminos en el vocabulario. Fue presentado
	por primera vez por Salton en 1971, en un art\'iculo llamado: 
	\begin{quotation}
	    \textit{A Vector Space Model for Automatic Indexing}.
	\end{quotation}
    En el cual presentaban el marco te\'orico para el modelo vectorial.\\

	\subsection{Modelo Probabil\'istico}
	Un modelo probabil\'istico o estad\'istico es la forma que pueden tomar un conjunto de datos obtenidos de muestreos de otros datos con comportamiento que se supone aleatorio. Un modelo estadístico es un tipo de modelo matemático que usa la probabilidad, y que incluye un conjunto de asunciones sobre la generación de algunos datos muestrales, de tal manera que asemejen a los datos de una población mayor. El modelo probabilístico como modelo de recuperación de independencia binaria fue desarrollado por Stephen Robertson y Spark Jones. Este modelo afirma que pueden caracterizarse los documentos de una colección mediante el uso de términos de indización.\\

	\subsection{Modelo Booleano Extendido}
	El Modelo Booleano Extendido fue presentado en un art\'iculo de 
	\begin{quotation}
		\textit{Communications of the ACM en el año 1983, por Gerard Salton, Edward A. Fox y Harry Wu}. 
	\end{quotation}
	El propósito del Modelo Booleano Extendido es superar las desventajas del Modelo Booleano que ha sido utilizado en recuperación de información. 
	El Modelo Booleano no considera los pesos de los términos en las consultas y el conjunto respuesta de una consulta booleana 
	es con frecuencia demasiado pequeño o demasiado grande. 
	La idea del modelo extendido es hacer uso de la correspondencia parcial y los pesos de los términos del Modelo de Espacio Vectorial, 
	combinándolos con las propiedades del Álgebra Booleana. De esta forma, un documento puede ser un tanto relevante 
	si contiene algunos términos de la consulta, y puede ser obtenido como respuesta, mientras que en el Modelo Booleano esto no ocurre.

	\section{Desarrollo}

	\subsection{Preprocesamiento}
	
	La primera etapa del sistema es el preprocesamiento. La principal caracter\'istica en este aspecto es que se hace un parse de las colecciones que tienen un formato especifico a objetos que podemos controlar mejor. Luego se eliminan las palabras $\it{stopwords}$ y se hace $\it{stemming}$ a las palabras resultantes, todo esto usando la biblioteca $\it{NLTK}$.
	
	\subsubsection{$default\_procesor$:}
	
	Este m\'etodo recibe como par\'ametro el texto a procesar en string y el lenguaje del texto. Se tokenizan las palabras, se aplica stemming y se eliminan las stopwords. Devuelve una lista con las palabras del texto normalizadas.
	
	\subsubsection{Documento:}
	
	Este objeto simboliza un documento de la colecci\'on. Y cuando se inicializa se le aplica al texto y al t\'itulo un procesador de texto. En este caso es $default\_procesor$, pero puede ser otro que se estime conveniente
	
	\subsubsection{Collection:}
	
	Representa la colecci\'on en s\'i despu\'es de aplicar el parser. Por defecto de este objeto heredan tres tipo de colecciones, CranCollection, ReutersCollection y NewsGroupCollection.\\
	
	Luego de tener las colecciones como listas de Documentos podemos usar los modelos implementados.

	\subsection{Modelo Vectorial}
	Funcionamiento del modelo vectorial:\\
	\begin{itemize}
		\item Se crea un vocabulario con todos los t\'erminos de la colecci\'on de documentos.
		\item Se crea un vector de pesos para cada documento de la colecci\'on de documentos.
		\item Se crea un vector de pesos para la consulta.
		\item Se calcula la similitud entre el vector de pesos de la consulta y el vector de pesos de cada documento de la colecci\'on de documentos.
	\end{itemize}

	Para la asiganci\'on de pesos a los t\'erminos de la consulta y de los documentos se us\'o la siguiente f\'ormula:\\
	\begin{equation}
		w_{i,q} =(a + (1-a)\frac{freq_{i,q}}{max_{l}freq_{l,q}}) \times{log\frac{N}{n_i}}
	\end{equation}
	Donde: \\
	\begin{itemize}
		\item $w_{i,q}$ es el peso del t\'ermino $i$ en el documento $q$.
	\end{itemize}

	Para el c\'alculo de la similitud entre el vector de pesos de la consulta y el vector de pesos de cada documento de la colecci\'on 
	de documentos se us\'o la siguiente f\'ormula:\\
	\begin{equation}
		sim(d_{j}, q) = \frac{\sum_{i} w_{i,j} \times w_{i,q}}{\sqrt{\sum_{i} w_{i,j}^{2}} \times \sqrt{\sum_{i} w_{i,q}^{2}}}
	\end{equation}

	Tambi\'en conocida como la \textit{Similitud de Cosenos}.\\
	Donde: \\
	\begin{itemize}
		\item $sim(d_{j}, q)$ es la similitud entre el documento $d_{j}$ y la consulta $q$.
		\item $w_{i,j}$ es el peso del t\'ermino $i$ en el documento $d_{j}$.
		\item $w_{i,q}$ es el peso del t\'ermino $i$ en la consulta $q$.
		\item $X$ es el producto escalar.
		\item $\sqrt{\sum_{i} w_{i,j}^{2}}$ es la norma del vector de pesos del documento $d_{j}$.
		\item $\sqrt{\sum_{i} w_{i,q}^{2}}$ es la norma del vector de pesos de la consulta $q$.
	\end{itemize}

	Como resultado de aplicar el modelo vectorial a una consulta se obtiene una lista de documentos ordenados de mayor a menor similitud. 
	Cuyo tama\~no de lista puede ser definido por el usuario.\\

	\subsection{Retroalimentaci\'on del Modelo Espacio Vectorial}
	
	La idea de la retroalimentaci\'on en este modelo se basa en encontrar un vector consulta $\overrightarrow{q}$ que maximice la similitud con los documentos relevantes y la minimice con los documentos no relevantes. Utilizando el coseno del \'angulo comprendido entre los vectores de la consulta y los documentos relevantes y no relevantes, se tiene que 
	
	\begin{equation}
	\overrightarrow{q}_{opt} = \frac{1}{|C_r|}\sum_{\overrightarrow{d_j} \in C_r} \overrightarrow{d_j} - \frac{1}{|C_{nr}|}\sum_{\overrightarrow{d_j} \in C_{nr}} \overrightarrow{d_j}
	\end{equation}
	
	Donde $C_r$ es el conjunto de los documentos relevantes y $C_{nr}$ es el conjunto de los documentos no relevantes.
		
	La consulta \'optima es el vector diferencia entre los centroides de los documentos relevantes y los no relevantes. Como en la pr\'actica solo se conoce parcialmente el conjunto de los documentos relevantes y no relevantes surge el algoritmo de Rocchio(1971)
	
	\begin{equation}
	\overrightarrow{q}_m = \alpha\overrightarrow{q}_0 + \frac{\beta}{|C_r|}\sum_{\overrightarrow{d_j} \in C_r} \overrightarrow{d_j} - \frac{\gamma}{|C_{nr}|}\sum_{\overrightarrow{d_j} \in C_{nr}} \overrightarrow{d_j}
	\end{equation}
	
	Donde $\alpha$,$\beta$ y $\gamma$ son pesos establecidos para cada t\'ermino de la consulta. De esta forma podemos ajustar a conveniencia la importancia de cada uno. Los valores comunmente utilizados son: $\alpha=1$,$\beta=0.75$ y $\gamma=0.15$

	\subsection{Modelo Probabil\'istico}
	El modelo probabilístico tiene como supuestos:\\
	\begin{itemize}
		\item Los documentos de una colección son independientes entre sí.
		\item Cada t\'ermino de un documento es independiente de los dem\'as t\'erminos.
		\item Los documentos de una colección son independientes de la consulta.
        \item Cada t\'ermino de la consulta es independiente de los dem\'as t\'erminos.
        \item El orden de los t\'erminos de la consulta no es relevante.
    \end{itemize}

	Para el modelo probabil\'istico los pesos de los t\'erminos en el documento presentan un estado binario, es decir,
	los t\'erminos que aparecen en el documento tienen un peso de 1 y los que no aparecen tienen un peso de 0.\\
	\\
	Para el c\'alculo de la similitud entre el vector de pesos de la consulta y el vector de pesos de cada documento de la colecci\'on
	de documentos se us\'o la siguiente f\'ormula:\\
	\begin{equation}
		sim(d_{j}, q) = \sum_{i} w_{i,j} * w_{i,q} * \log{\frac{p_{i}(1 - r_{i})}{r_{i}(1 - p_{i})}}
	\end{equation}
	\\
	Donde: \\
	\begin{itemize}
		\item $sim(d_{j}, q)$ es la similitud entre el documento $d_{j}$ y la consulta $q$.
		\item $w_{i,j}$ es el peso del t\'ermino $i$ en el documento $d_{j}$.
		\item $w_{i,q}$ es el peso del t\'ermino $i$ en la consulta $q$.
		\item $p_{i}$ es la probabilidad de que el t\'ermino $i$ aparezca en un documento relevante de la colecci\'on.
		\item $r_{i}$ es la probabilidad de que el t\'ermino $i$ aparezca en un documento no relevante de la colecci\'on.
	\end{itemize}

	Lo que es un despeje de la f\'ormula:\\
	\begin{equation}
		O(R|\vec{d}) = \frac{p(R|\vec{d})}{p(\bar{R}|\vec{d})}
    \end{equation}
	\\
	Donde: \\
	\begin{itemize}
		\item $O(R|\vec{d})$ es la raz\'on de probabilidad entre la relevancia y la no relevancia del documento $\vec{d}$.
		\item $p(R|\vec{d})$ es la probabilidad de que el documento $\vec{d}$ sea relevante.
		\item $p(\bar{R}|\vec{d})$ es la probabilidad de que el documento $\vec{d}$ no sea relevante.
    \end{itemize}


	Como resultado de aplicar el modelo probabil\'istico a una consulta se obtiene una lista de documentos ordenados de mayor a menor similitud.
	Cuyo tama\~no de lista puede ser definido por el usuario.\\

	\subsection{Retroalimentaci\'on del Modelo Probabil\'istico}
	La idea tras la Retroalimentaci\'on del Modelo Probabil\'istico es recalcular la probabilidad de
	relevancia de los t\'erminos de la consulta, basado en los documentos que fueron relevantes en la
	b\'usqueda anterior.\\
	C\'omo as\'i tambi\'en recalcular la probabilidad de no relevancia de los t\'erminos de la consulta,
	basado en los documentos que fueron no relevantes en la b\'usqueda anterior.\\
	Que es de igual manera la forma para recalcular el peso de relevancia de un t\'ermino en en un documento dado 
	que el mismo aparece en la consulta.\\
	Esta idea de Retroalimentaci\'on est\'a dada por la f\'ormula:\\
	\begin{equation}
		p_{i}^{(2)} = \frac{|V_{i}| + k * p_{i}^{(1)}}{|V| + k}
	\end{equation}
	\\
	Donde: \\
	\begin{itemize}
		\item $p_{i}^{(2)}$ es la probabilidad de que el t\'ermino $i$ aparezca en un documento relevante de la colecci\'on.
		\item $V_{i}$ es el conjunto de documentos relevantes que contienen el t\'ermino $i$.
		\item $k$ es un par\'ametro de suavizado.
		\item $p_{i}^{(1)}$ es el valor anterior de probabilidad de que el t\'ermino $i$ aparezca en un documento relevante de la colecci\'on.
		\item $V$ es el conjunto de documentos relevantes.
	\end{itemize}

    
	Mientras que la no relevancia de un t\'ermino en un documento dado que el mismo aparece en la consulta se calcula de la siguiente manera:\\
	\begin{equation}
		r_{i} = \frac{n_{i} - |V_{i}|}{N - |V|}
	\end{equation}
	\\
	Donde: \\
	\begin{itemize}
		\item $r_{i}$ es la probabilidad de que el t\'ermino $i$ aparezca en un documento no relevante de la colecci\'on.
		\item $n_{i}$ es el n\'umero de documentos que contienen el t\'ermino $i$.
		\item $V_{i}$ es el conjunto de documentos relevantes que contienen el t\'ermino $i$.
		\item $N$ es el n\'umero total de documentos en la colecci\'on.
		\item $V$ es el conjunto de documentos relevantes.
	\end{itemize}

	\subsection{Modelo Booleano Extendido}
	El Modelo Booleano Extendido es una extensi\'on del Modelo Booleano, en el cual se considera la
	el peso de relevancia de un t\'ermino en un documento.\\
	Este peso est\'a dado por la f\'ormula:\\
	\begin{equation}
		w_{i,j} = tf_{i,j}*\frac{Idf_{x}}{max_{i}Idf_{x}}
	\end{equation}
	\\
	Donde: \\
	\begin{itemize}
		\item $w_{i,j}$ es el peso del t\'ermino $i$ en el documento $d_{j}$.
		\item $tf_{i,j}$ es la frecuencia del t\'ermino $i$ en el documento $d_{j}$.
		\item $Idf_{x}$ es el inverso de la frecuencia del t\'ermino $i$ en la colecci\'on.
		\item $max_{i}Idf_{x}$ es el m\'aximo valor de $Idf_{x}$.
	\end{itemize}
	
	\paragraph*{Query:}
	La consulta para este modelo se asume que ser\'a proveida en una forma normal disyuntiva, sin p\'erdida de la generalidad,
	dado que toda expresi\'on del \'algebra booleana puede ser expresada como tal.\\

	\paragraph*{C\'alculo de similitud:}
	La similitud entre una consulta y un documento se calcula de la siguientes maneras:\\
	\begin{equation}
		sim(q_{or},d_{j}) = \sqrt{\frac{\sum_{i}w_{i}^{2}}{t}}
    \end{equation}
	\\
	Esto para el caso donde la consulta es una expresi\'on normal disyuntiva.\\
	\\
	\begin{equation}
		sim(q_{and},d_{j}) = 1 - \sqrt{\frac{\sum_{i}(1 - w_{i})^{2}}{t}}
	\end{equation}
	\\
	Esto para el caso donde la consulta es una expresi\'on normal conjuntiva.\\
	\\
	Donde: \\
	\begin{itemize}
		\item $w_{i}$ es el peso del t\'ermino $i$ en el documento $d_{j}$.
		\item $t$ es el n\'umero de t\'erminos en la consulta.
		\item $q_{or}$ es la consulta en forma normal disyuntiva.
		\item $q_{and}$ es la consulta en forma normal conjuntiva.
	\end{itemize}

	\subsection{¿Por qu\'e estos modelos?}
	La idea de usar estos modelos es la simplicidad de su comprensi\'on para el lector, los buenos resultados que pueden traer 
	a cabo a pesar de su simplicidad y la facilidad de implementaci\'on de los mismos.\\

	El modelo de Espacio Vectorial es de los modelos cl\'asicos el de mejores resultados hasta la fecha,
	tiene un conjunto de ventajas que lo hacen muy atractivo para su uso en la recuperaci\'on de informaci\'on.\\
	Entre las ventajas que tiene este modelo se encuentran:\\
	\begin{itemize}
		\item Es un modelo de recuperaci\'on de informaci\'on que se basa en la similitud sem\'antica entre documentos y consultas.
		\item Tiene como resultado un ranking de relevancia.
		\item Tiene en cuenta la frecuencia de los t\'erminos en los documentos.
		\item Calcula el peso de un t\'ermino en un documento de una manera m\'as precisa que el modelo booleano e incluso para juicio de muchos m\'as precisa que el modelo probabil\'istico.
		\item La forma de calcular la similitud es muy usada en muchas otras esferas con excelentes resulatados. Adem\'as dada la modelaci\'on que presenta al problema tiene mucho sentido de aplicarla.
		\item No se basa solamente en una coincidencia total del documeno y la consulta, como el modelo booleano, sino tambi\'en en una parcial.
	\end{itemize}

	El modelo Probabi\'istico ha alcanzado muy buenos resultados tambi\'en a traves de los a\~nos. 
	Mejorando en gran medida al modelo booleano en cuestiones generales, 
	aunque tambi\'en tiene sus desventajas se decidi\'o usarlo como comparativa a los otros modelos y la capacidad de mejora que presenta con la 
	retroalimentaci\'on de los resultados.\\

	Entre las ventajas que tiene este modelo se encuentran:\\
	\begin{itemize}
		\item Muestra un ranking de relevancia.
		\item Capacidad de de mejora con la retroalimentaci\'on de los resultados.
	\end{itemize}
    
	El Modelo Boleano Extendido es una extensi\'on del Modelo Booleano, en el cual se considera tiene
	los mejores resultados hasta el momento, incluso mejorando el \textit{Fuzzy}, lo cual es un buen indicador de su eficacia.\\

	Entre las ventajas que tiene este modelo se encuentran:\\
	\begin{itemize}
		\item Muestra un ranking de relevancia.
		\item Analiza el peso cada t\'ermino por su frecuencia en el documento y la frecuencia inversa en la colecci\'on de documentos.
		\item Brinda una mejor aproximaci\'on a la similitud sem\'antica entre documentos y consultas.
		\item Tiene muy buena claridad de que se desea obtener como resultado.
	\end{itemize}

	\section{Evaluaci\'on}
	
    Para al evaluaci\'on de los modelos se utiliz\'o la colecci\'on de documentos \textit{cranfield}, 
	dado que la misma presentaba ejemplos de querys con su respectiva relevancia de cada documento. 
	En el caso del modelo \textit{Boleano Extendido} no se pudo realizar la comprobaci\'on dada la diferencia de representar la pregunta.\\

	\begin{table}[]
		\begin{tabular}{lllll}
		Modelo	&  Precisi\'on&  Recobrado & F1 \\
		Vectorial &	0.147 & 0.001 & 0.002 & \\
		Probabil\'istico & 0.15 & 0.003 & 0.005 &  \\
		\end{tabular}
	\end{table}

	\section{Forma de uso}
	
	Para usar la aplicaci\'on solo debe ejecutar con python el archivo \textit{main.py}. Tambien debe crear dos carpetas en el directorio ra\'iz: data y models, esto se utiliza para guardar las colecciones y los modelos respectivamente en forma de serializado para optimizar el tiempo de ejecuci\'on
	
	\subsection{Requerimientos}
	
	Debe tener instalado python$\geq$3.7 e instalar los modulos en \textit{requirements.txt} ejecutando el comando \textit{pip install -r requirements.txt}. Por \'ultimo se deben descargar las palabras necesarias para ejecutar el modulo $NLTK$, para esto ejecutamos el siguiente c\'odigo de python
	
	\begin{python}
	import nltk
	
	nltk.download('punkt')
	nltk.download('stopwords')
	\end{python}
	

    \subsection{Colecciones}
    
	Las colecciones deben estar dentro de una carpeta llamada corpus, dentro del directorio ra\'iz y dentro debe haber una carpeta para cada tipo de colecci\'on, por ahora solo se leen las tres explicadas en el preprocesamiento

	\section{Conclusiones}
	A modo de Conclusiones se puede decir que la implementaci\'on actual del modelo Probabil\'istico es el que mejor resultados entre los vistos en el proyecto, 
	se nota la influencia del uso de retroalimentaci\'on del \textit{Modelo Probabil\'istico} y en el \textit{Espacio Vectorial}. \\
	Los resultados obtenidos dada la implementaci\'on de cada uno de los modelos abordados en el proyecto han sido satisfactorios, permitiendo ahondar
	en el conocimiento de los mismos y su aplicaci\'on en la recuperaci\'on de informaci\'on.\\

	Una cuesti\'on del Modelo Booleano Extendido a resaltar es el hecho de que no tiene retroalimetaci\'on implementada, por el hecho de que
	la retroalimentaci\'on de este modelo, que se pudo investigar es muy similar a la del modelo probabil\'istico, donde a juicio de los autores, 
	no se ajusta a la forma de an\'alisis del modelo, mientras que otras tienen un muy complejo desarrollo,
	debido a que requieren entrenamiento de \textit{Redes Neuronales}, por lo que se decidi\'o no implementarla.\\
	\\

	\paragraph*{Ventajas:}
	\begin{itemize}
		\item El proyecto tiene una buena extensibilidad en cuanto a parsear las colecciones nuevas.
		\item Presenta bastante seguridad en la forma en la que se guarda la retrolaimentaci\'on y las colecciones para cada modelo.
		\item La respuesta de la aplicaci\'on es bastante r\'apida.
		\item La interfaz es bastante intuitiva.
	\end{itemize}

	\paragraph*{Desventajas:}
	\begin{itemize}
		\item La interfaz no es muy atractiva.
		\item No se puede realizar la retroalimentaci\'on del modelo booleano extendido.
		\item El modelo probabil\'istico guarda la informaci\'on en forma de texto plano.
    \end{itemize}

	\section{Recomendaciones}
	Se recomienda para futuros avances en este proyecto la revisi\'on de otras formas de retroaliementaci\'on para el modelo de Espacio Vectorial
	y el Modelo Probabil\'istico, e incluso ver algunas de las formas de implementar retroalimentaci\'on con Machine Learning, en este caso tambi\'en 
	para el Modelo Booleano Extendido.\\
	\\

	Mejorar la interfaz visual del programa, para que sea m\'as atractiva y de mejor comprensi\'on para el usuario.\\
	\\

	Hacer una base de datos para que al aplicaci\'on no necesite cargar las colecciones 
	enteras cada vez que se quiera iniciar el programa y as\'i mejorar la eficiencia.\\

	Realizar un an\'alisis m\'as profundo de la precisi\'on de los modelos, con ello incluir la evaluaci\'on del Modelo Booleano Extendido.\\

	\section*{Bibliograf\'ia}
	\begin{itemize}
		\item Conferencias del curso de Sistemas d recuperaci\'on de informaci\'on.
		\item Wikipedia.
		\item Otros Art\'iculos de Internet.
	\end{itemize}

\end{document}